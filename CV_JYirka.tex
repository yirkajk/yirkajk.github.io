\documentclass[11pt,letterpaper,serif]{moderncv}
\usepackage[margin=1in]{geometry}
\moderncvstyle{banking}
\moderncvcolor{black}

\usepackage[T1]{fontenc}

\usepackage{fontawesome5}

\usepackage{amssymb}
\usepackage{MnSymbol}

\urlstyle{same} % Forces URLs to use same font, rather than a monospace font
\usepackage{changepage} % For indenting blocks of text

\name{Justin}{Yirka} % Unused, since we make our own header

\usepackage{lastpage} % for the Footer
\fancypagestyle{firstpage}{\lfoot{Curriculum vitae}\rfoot{\today \quad \thepage / \pageref{LastPage}}}
\fancypagestyle{normal}{\rfoot{J. Yirka \quad \thepage / \pageref{LastPage}}}

\renewcommand*{\subsectionfont}{\large\upshape\fontseries{b}\selectfont} % Overrides font for subsection headers. Make them bold.

\usepackage{xpatch} % To remove default formatting from cventry.
% This is from moderncvbodyiii.sty.
\xpatchcmd{\cventry}{\itshape #2}{#2}{}{} % year
\xpatchcmd{\cventry}{\itshape #3}{#3}{}{} % "title"
\xpatchcmd{\cventry}{\bfseries #4}{#4}{}{} % "institution"
\xpatchcmd{\cventry}{\bfseries #5}{#5}{}{} % city
% #6 "grade" entry is covered by fix to #3 "title"

\usepackage{enumitem} % for defining itemsep as needed. Default is 0.
\setlist{itemsep=0em} % default
\newcommand{\pubItemSep}{0em}

%=========================================================================================
%  BEGIN DOCUMENT
%=========================================================================================
\begin{document}

\pagestyle{normal}

\hypersetup{pdftitle={CV Justin Yirka}, pdfauthor={Justin Yirka}}

\thispagestyle{firstpage}

\begin{center}
	{\LARGE\textbf{Justin Yirka}}

	\smallskip{}

	Ph.D. Candidate in Computer Science, Graduating in 2025

	\medskip{}

	\href{mailto:yirka@utexas.edu}{yirka@utexas.edu}
	\qquad
	703-229-7956

	\href{https://www.justinyirka.com}{JustinYirka.com}

	\medskip{}

	\href{https://arxiv.org/a/yirka_j_1.html}{arXiv.org/a/yirka\_j\_1.html}

	\faLinkedin{} \href{https://www.linkedin.com/in/justinyirka/}{linkedin.com/in/justinyirka}

	\faGraduationCap{} \href{https://scholar.google.com/citations?user=UxIpR_UAAAAJ}{scholar.google.com/citations?user=UxIpR\_UAAAAJ}

	\faYoutube{}
	\href{https://www.youtube.com/playlist?list=PLHxZKg_X23Knp1fhJJI2u9HZP39uhbr7O}{youtube.com/@JustinYirka/playlists}
\end{center}


%%% Increase vertical spacing after paragraphs in remainder of document (default parskip in moderncv is 0em) %%%
\setlength{\parskip}{0em}
% itemsep is defined separately

%=========================================================================================
%  SECTION: Research Interests
%=========================================================================================
\section{Research Interests}
Quantum computing and Theoretical computer science

\qquad Computational complexity theory, Hamiltonian complexity, Quantum algorithms


%=========================================================================================
%  SECTION: Education
%=========================================================================================
\section{Education}


\cventry{Expected May 2025}{\textbf{Ph.D. in Computer Science} | The University of Texas at Austin (UT)}{}{}{}
{
		\normalsize
		\begin{adjustwidth}{.25in}{}
			Advised by Scott Aaronson
		\end{adjustwidth}
}

\cventry{2022}{\textbf{M.S. in Computer Science} | The University of Texas at Austin}{}{}
{}{}

\smallskip

\cventry{2018}{\textbf{B.S. in Computer Science} | Virginia Commonwealth University (VCU)}{}{}{}{}
\vspace{-\parsep}
\cventry{Concurrent degrees}{\textbf{B.S. in Mathematical Sciences}}{}{}{}
{
	\begin{adjustwidth}{.25in}{}
		Specialization in Data Science \&
		Concentration in Pure Math \newline
		Minor in Physics \newline
		University Honors
	\end{adjustwidth}
}

%%%%%%%%%%%%%%%%%%%%%%%%%%%%%%%%%%%%%%%%%%%%%%%%%%%%%%%%%%%%%%%%%%%
%
%%%%%%%%%%%%%%%%%%%%%%%%%%%%%%%%%%%%%%%%%%%%%%%%%%%%%%%%%%%%%%%%%%%
\section{Research Positions}
\cventry{June 2023--present}{\textbf{R\&D Intern} | Sandia National Laboratories}{}{}{}
{
	\begin{adjustwidth}{.25in}{}
		Advised by Ojas Parekh and John Kallaugher \newline
		Topic: Hardness of estimating optimal product states of local Hamiltonians. Quantum Max-Cut, Vector Max-Cut, and Quantum constrained optimization problems.
		Alternative query models.
	\end{adjustwidth}
}

\cventry{Summer 2019}{\textbf{Summer School Fellow} | Los Alamos National Laboratories}{}{}{}
{
	\begin{adjustwidth}{.25in}{}
		Advised by Yi\u{g}it Suba\c{s}\i \newline
		Topic: Near-term (NISQ) quantum algorithms. Studied use of mid-circuit measurements and resets to construct circuits for entanglement spectroscopy which were noise-resilient \textit{and} low-width. \newline
		Implemented noisy simulations with Qiskit, Python, Unix, Jupyter. Managed project with git. Tested algorithms on Honeywell quantum hardware.
	\end{adjustwidth}
}

\cventry{Summer 2018}{\textbf{Research Assistant} | Graph Theory Computational Discovery Lab, VCU}{}{}{}
{
	\begin{adjustwidth}{.25in}{}
		Supervised by Craig Larson \newline
		Topic: Automated conjecturing software applied to graph theory.\newline
		Maintained database of graphs, their properties, and known theorems. Managed open-source project and programmed using git, GitHub, and Sage/Python.
	\end{adjustwidth}
}

\cventry{Summer 2017}{\textbf{NSF REU Researcher} | QuICS, University of Maryland}
{}{}{}
{
	\begin{adjustwidth}{.25in}{}
		Advised by Andrew Childs, Jianxin Chen, and Amir Kalev \newline
		Topic: Quantum tomography. Investigated minimum number of Pauli observables necessary to identify a quantum pure state.
	\end{adjustwidth}
}

\cventry{2015--2016}{\textbf{Research Assistant} | Quantum Computing Lab, VCU}{}{}{}
{
	\begin{adjustwidth}{.25in}{}
		Advised by Sevag Gharibian \newline
		Topic: Complexity theory. Studied quantum oracle classes  (e.g. {\scriptsize $\textup{P}^{\textup{QMA[log]}}$}) and Hamiltonian complexity. Helped develop a ``quantum PH'' and ``quantum Toda's Theorem'' ({\scriptsize $\textup{QCPH}\subseteq \textup{P}^{\textup{PP}^{\textup{PP}}}$}).
	\end{adjustwidth}
}

%=========================================================================================
%  SECTION: Research papers
%=========================================================================================
\section{Research Papers and Talks}
\vspace{-.1in}
{\footnotesize
	Authors are listed alphabetically, as is standard in TCS, unless marked *.

	Some conference talks are accompanied by published proceedings.
	Filled labels \textbullet{} indicate I gave the talk.

	Visit my website for links to recordings, slide pdfs, etc.

	~
}

J. Yirka.\quad
Even quantum advice is unlikely to solve \textup{PP}.
\begin{adjustwidth}{.1in}{}
	\begin{itemize}
		\item[--] Preprint. \href{https://arxiv.org/abs/2403.09994}{arXiv:2403.09994} and \href{https://eccc.weizmann.ac.il/report/2024/052/}{ECCC:TR24-052}, 2024.
	\end{itemize}
\end{adjustwidth}
\medskip

S. Grewal and J. Yirka.\quad
The Entangled Quantum Polynomial Hierarchy Collapses.
\begin{adjustwidth}{.1in}{}
	\begin{itemize}
		\item Proceedings of \textit{39th Computational Complexity Conference (CCC)}, Ann Arbor, USA, 2024.\quad
		\\
		\href{https://doi.org/10.4230/LIPIcs.CCC.2024.6}{doi:10.4230/LIPIcs.CCC.2024.6}.
		\item[--] \href{https://arxiv.org/abs/2401.01453}{arXiv:2401.01453} and \href{https://eccc.weizmann.ac.il/report/2024/006/}{ECCC:TR24-006}, 2024.
	\end{itemize}
\end{adjustwidth}
\medskip

J. Kallaugher, O. Parekh, K. Thompson, Y. Wang, and J. Yirka.\quad
Complexity Classification of Product State Problems for Local Hamiltonians.
\begin{adjustwidth}{.1in}{}
	\begin{itemize}
		\item[\textbullet] Contributed talk at Innovations in Theoretical Computer Science conference (ITCS), New York, USA, 2025.\quad
		\href{https://doi.org/10.4230/LIPIcs.ITCS.2025.63}{doi:10.4230/LIPIcs.ITCS.2025.63}.
		\item[\textbullet] Contributed talk at Conference on Quantum Information Processing (QIP), Taipei, Taiwan, 2024.
		\item[--] \href{https://arxiv.org/abs/2401.06725}{arXiv:2401.06725}, 2024.
	\end{itemize}
\end{adjustwidth}
\medskip

J. Yirka and Y. Subasi.*\quad
Qubit-efficient entanglement spectroscopy using qubit resets.
\begin{adjustwidth}{.1in}{}
	\begin{itemize}
		\item[--] \textit{Quantum}, 5:535, 2021.\quad
		\href{https://doi.org/10.22331/q-2021-09-02-535}{doi:10.22331/q-2021-09-02-535}.
		\item[\textbullet] Contributed talk by J. Yirka at Conference for Young Quantum Information Scientists (YQIS), Virtual, 2021.
		\item Contributed talk at APS March Meeting, Virtual, 2021.
		\item[\textbullet] Contributed talk at 20th Asian Quantum Information Science Conference (AQIS), Virtual, 2020.
		\item[--] \href{https://arxiv.org/abs/2010.03080}{arXiv:2010.03080}, 2020.
	\end{itemize}
\end{adjustwidth}
\medskip

S. Gharibian, S. Piddock, and J. Yirka.\quad
Oracle complexity classes and local measurements on physical Hamiltonians.
\begin{adjustwidth}{.1in}{}
	\begin{itemize}
		\item In Proceedings of \textit{37th Symposium on Theoretical Aspects of Computer Science (STACS)}, Montpellier, France, 2020.\quad
		\href{https://doi.org/10.4230/LIPIcs.STACS.2020.20}{doi:10.4230/LIPIcs.STACS.2020.20}.
		\item[\textbullet] Contributed talk at Conference on Quantum Information Processing (QIP), Shenzhen, China, 2020.
		\item[\textbullet] Contributed talk at Asian Quantum Information Science Conference (AQIS), Nagoya, Japan, 2018.
		\item[--]  \href{https://arxiv.org/abs/1909.05981}{arXiv:1909.05981}, 2019.
	\end{itemize}
\end{adjustwidth}
\medskip

S. Gharibian, M. Santha, J. Sikora, A. Sundaram, and J. Yirka.\quad
Quantum generalizations of the polynomial hierarchy with applications to QMA(2).
\begin{adjustwidth}{.1in}{}
	\begin{itemize}
		\item[--] \textit{Computational Complexity}, 31:12, 2022.\quad
		\href{https://doi.org/10.1007/s00037-022-00231-8}{doi:10.1007/s00037-022-00231-8}.
		\item Contributed talk at Asian Quantum Information Science Conference (AQIS), Nagoya, Japan, 2018. --- \textbf{``Long''/plenary talk: top 7\% of submissions}.
		\item In Proceedings of \textit{43rd Symposium on Mathematical Foundations of Computer Science (MFCS)}, Liverpool, UK, 2018. \quad
		\href{https://doi.org/10.4230/LIPIcs.MFCS.2018.58}{doi:10.4230/LIPIcs.MFCS.2018.58}.
		\item[--] \href{https://arxiv.org/abs/1805.11139}{arXiv:1805.11139}, 2018.
	\end{itemize}
\end{adjustwidth}
\medskip

S. Gharibian and J. Yirka.\quad
The complexity of simulating local measurements on quantum systems.
\begin{adjustwidth}{.1in}{}
	\begin{itemize}[itemsep=\pubItemSep]
		\item[--] \textit{Quantum}, 3:189, 2019. \quad
		\href{https://doi.org/10.22331/q-2019-09-30-189}{doi:10.22331/q-2019-09-30-189}.
		\item In Proceedings of \textit{12th Conference on the Theory of Quantum Computation, Communication,
		and Cryptography (TQC)}, Paris, France, 2017. \quad
		\href{https://doi.org/10.4230/LIPIcs.TQC.2017.2}{doi:10.4230/LIPIcs.TQC.2017.2}.
		\item[--] \href{https://arxiv.org/abs/1606.05626}{arXiv:1606.05626}, 2016.
	\end{itemize}
\end{adjustwidth}

%=========================================================================================
%  SECTION: Research Experience
%=========================================================================================
\section{Other Research Experience}

\subsection{Non-quantum computing work}
N. Bushaw, V. Gupta, C. Larson, S. Loeb, M. Norge, J. Parrish, N. Van Cleemput, J. Yirka, and G. Wu.\quad
New conditions for graph Hamiltonicity
\begin{adjustwidth}{.1in}{}
	\begin{itemize}[itemsep=\pubItemSep]
		\item[--] \textit{Involve, a Journal of Mathematics}, 18(1):79--89, 2025.\quad
		\href{https://doi.org/10.2140/involve.2025.18.79}{10.2140/involve.2025.18.79}.
	\end{itemize}
\end{adjustwidth}
\medskip

J. Yirka.\quad
Evaluation of TCP header fields for data overhead efficiency.
\begin{adjustwidth}{.1in}{}
	\begin{itemize}[itemsep=\pubItemSep]
		\item[$\filledtriangleright$] Poster at National Conference on Undergraduate Research (NCUR), Asheville, NC, USA, 2016.
		\item[$\filledtriangleright$] Poster at VCU Symposium for Undergraduate Research and Creativity, Richmond, VA, USA, 2015. --- \textbf{Awarded ``Launch Award for Outstanding Research Poster''}
	\end{itemize}
\end{adjustwidth}

%==========  Workshops  =======================================================
\smallskip
\subsection{Workshops and Visits}

\cventry{}{Albuquerque, USA}{All-hands meeting | Quantum Systems Accelerator, a DOE Research Center}{June 2021}{}{}

\cventry{}{Quantum Complexity: Quantum PCP, Area Laws, and Quantum Gravity}{Workshop | Simons Institute for the Theory of Computing. Berkeley, USA.}{March 2024}{}{}

\cventry{}{Quantum Complexity: Theory and Application}{Invited Workshop | Schloss Dagstuhl. Virtual.}{June 2021}{}{}

\cventry{}{Collaboration with Sevag Gharibian}{Visiting Researcher | University of Paderborn. Germany.}{November 2018}{}
{
	\begin{adjustwidth}{.25in}{}
		Topic: Complexity theory and algorithms. Studied $\textup{QMA}_{1}$-hardness of the quantum satisfaction problem ($k$-QSAT) given qudits of lower dimensions.
	\end{adjustwidth}
}


%%==========  Posters  =======================================================
\smallskip
\subsection{Posters}
%\vspace{-.1in}
{\footnotesize
	Filled labels $\blacktriangleright$ indicate I presented the poster.
}
\begin{itemize}
	\item[$\blacktriangleright$]
	J. Kallaugher, O. Parekh, K. Thompson, Y. Wang, and J. Yirka.\quad
	Complexity Classification of Product State Problems for Local Hamiltonians.\quad
	DOE Quantum Systems Accelerator All-Hands meeting. Albuquerque, USA, 2024.

	\item[$\blacktriangleright$]
	J. Kallaugher, O. Parekh, K. Thompson, Y. Wang, and J. Yirka.\quad
	Complexity Classification of Product State Problems for Local Hamiltonians.\quad
	Sandia Quantum Information Development Networking Day. Sandia National Laboratories, Albuquerque, USA, 2024.

	\item[$\blacktriangleright$]
	S. Grewal and J. Yirka.\quad
	The Entangled Quantum Polynomial Hierarchy Collapses.\quad
	Conference on Quantum Information Processing (QIP), Taipei, Taiwan, 2024.

	\item[$\smalltriangleright$]
	S. Gharibian, S. Piddock, and J. Yirka.\quad
	Oracle complexity classes and local measurements on physical Hamiltonians.\quad
	Conference on the Theory of Quantum Computation, Communication, and Cryptography (TQC), College Park, MD, USA, 2019.

	\item[$\blacktriangleright$]
	S. Gharibian, S. Piddock, and J. Yirka.\quad
	Oracle complexity classes and local measurements on physical Hamiltonians.\quad
	Workshop on Quantum Computing Theory in Practice (QCTIP), Bristol, UK, 2019.

	\item[$\filledtriangleright$]
	S. Gharibian, S. Piddock, and J. Yirka.\quad
	Oracle complexity classes and local measurements on physical Hamiltonians.\quad
	Conference on Quantum Information Processing (QIP), Boulder, CO, USA, 2019.

	\item[$\smalltriangleright$]
	S. Gharibian, M. Santha, J. Sikora, A. Sundaram, and J. Yirka.\quad
	Quantum generalizations of the polynomial hierarchy with applications to QMA(2).\quad
	Conference on Quantum Information Processing (QIP), Boulder, CO, USA, 2019.

	\item[$\filledtriangleright$]
	S. Gharibian and J. Yirka.\quad
	The complexity of simulating local measurements on quantum systems.\quad
	Conference on Quantum Information Processing (QIP). Seattle, USA, 2017.
\end{itemize}


%%==========  Departmental Seminars  =======================================================
\subsection{Seminars}

\begin{itemize}
	\item[\textbullet] PhD Proposal.\quad
	UT Department of Computer Science, 2024.

	\item[\textbullet] PhD Qualifying Exam talk (RPE).\quad
	UT Department of Computer Science, 2024.

	\item[\textbullet] Intro to Quantum Hamiltonians with old, new classical, and open questions.\quad
	UT theory student seminar, 2023.

	\item[\textbullet] Pure state tomography with Pauli observables.\quad
	QuICS, University of Maryland, 2017.

	\item[\textbullet] Quantum complexity of estimating local physical quantities.\quad
	VCU Department of Computer Science, 2016.\quad
	(Only undergraduate invited in previous 5 years.)
\end{itemize}


%=========================================================================================
%  SECTION: Teaching Experience
%=========================================================================================
\section{Teaching Positions}
%==========  UT  =======================================================

\cventry{}{Quantum Information Science (Web-based for M.S. program) {\footnotesize (CS 388Q)}}{\textbf{Head Teaching Assistant} | UT}{Spring 2022, 2023, 2024}{}
{
	\begin{adjustwidth}{.25in}{}
		Adapted and led entire course except for pre-recorded lectures.\\
		I was responsible for all other content and logistics, handling office hours, student concerns, academic integrity, and final grades nearly autonomously.
		Supervised 4 other teaching assistants.\\
		Spring 2022: 200 students, 1500 discussion board posts. Course evaluation 4.1~/~5.
		\\ Spring 2024: Course evaluation 4.91~/~5.
	\end{adjustwidth}
}

\smallskip

\cventry{}{Introduction to Quantum Information Science (Honors course) {\footnotesize (CS 358H)}}{\textbf{Teaching Assistant} | UT}{Fall 2021}{}
{
	\begin{adjustwidth}{.25in}{}
	With Scott Aaronson. Taught recitation and graded assignments.
	\end{adjustwidth}
}

\smallskip

\cventry{}{Introduction to Software Engineering (Java)}{\textbf{Instructor} | UT International Academy}{Summer 2021}{}
{
	\begin{adjustwidth}{.25in}{}
		Virtual.
		%Introductory course for international undergraduate students.
		Developed entire course including lectures and assignments.
		Course evaluation 4.88~/~5.
	\end{adjustwidth}
}

\smallskip

\cventry{}{Algebra with Applications {\footnotesize (MATH 141)}}{\textbf{Teaching Assistant} | VCU}{(2 semesters) 2016--2017}{}
{
	\begin{adjustwidth}{.25in}{}
		Assisted with daily in-class exercises, offered tutorials, graded assignments. \newline
		Average student evaluation scores --- Fall 2016: 4.78 / 5.0; Spring 2017: 4.36 / 5.0.
	\end{adjustwidth}
}

\smallskip

\cventry{}{CPR and first-aid courses for lifeguards}{\textbf{Instructor} | Department of Parks and Recreation, Prince William County, VA}{2016--2018}{}{}

\cventry{}{Honors Rhetoric {\footnotesize (HONR 200)} --- first-year honors writing and research course}{\textbf{Teaching Assistant} | VCU}{Fall 2015}{}{}


%=========================================================================================
%  SECTION: Funding
%=========================================================================================
\section{Scholarships and Funding \hfill{\footnotesize (all dollar amounts in USD)}}

\cventry{}{\$10,000}{\textbf{PI} | Quantum seminar series at UT}{2024--2025}{NSF CIQC}{}

\cventry{}{\$1,900}{Grants for seminar series by VCU RamDev software development club}{2016--2018}{VCU Student Government Association}{}

\cventry{}{\$660}{Mark A. Sternheimer Capstone Design Award}{2017}{VCU School of Engineering}
{
	\begin{adjustwidth}{.25in}{}
		Grant for developing and testing senior project app: Android, iOS, RasberryPi, AWS, Bluetooth LE.
	\end{adjustwidth}
}

\cventry{}{\$110,000}{VCU Presidential Scholarship}{2014--2018}{Virginia Commonwealth University}
{
	\begin{adjustwidth}{.25in}{}
		Awarded to 0.6\% of admitted students.\\
		Full cost of 4-year tuition, room, and board.
	\end{adjustwidth}
}

\cventry{}{\$80,000}{WPI Presidential Scholarship \textmd{[declined]}}{2014}{Worcester Polytechnic Institute}{}

\cventry{}{\$100,000}{Rensselaer Medal Merit Scholarship \textmd{[declined]}} {2014}{Rensselaer Polytechnic Institute}{}

%%%%%%%%%%%%%%%%%%%%%%%%%%%%%%%%%%%%%%%%%%%%%%%%%%%%%%%%%%%%%%%%%%%%%%%%%%%%%%%%%%%%%
\subsection{Travel grants}

\begin{itemize}
	\item \$600 for CCC 2024 in Ann Arbor, USA. CCC travel allowance / NSF.
	\item \$1,425 for Simons Institute workshop in Berkeley, CA, USA. CIQC (an NSF Quantum Challenge Institute), 2024.
	\item \$500 for QIP 2024 in Taipei, Taiwan. UT Graduate School.
	\item \$1,600 for QIP 2024 in Taipei, Taiwan. QIP student stipend.
	\item \$1,100 for QIP 2020 in Shenzhen, China. QIP student support / NSF.
	\item \$400 for QIP 2019 in Boulder, CO, USA. QIP student support / NSF.
	\item \$500 for QIP 2017 in Seattle, USA. VCU Honors College.
\end{itemize}

%\cventry{}{\$550}{Travel grant to present at NCUR 2016 in Asheville, NC, USA}{2016}{VCU Honors College}{}



%=========================================================================================
%  SECTION: Awards and Honors
%=========================================================================================
\section{Awards}

\cventry{2019, 2020}{\textbf{Honorable Mention} | NSF Graduate Research Fellowship Program (NSF GRFP)}{}{}{}
{
	\begin{adjustwidth}{.25in}{}
		Awarded twice. Granted to top 30\% of over 12,000 applicants.
	\end{adjustwidth}
}

\cventry{May 2018}{\textbf{Pure Mathematics Award} | VCU College of Humanities and Sciences}{}{}{}
{
	\begin{adjustwidth}{.25in}{}
		Student in pure math concentration with highest graduating GPA.
	\end{adjustwidth}
}

\cventry{2015}{\textbf{University Student Scholar Award} | Virginia Commonwealth University}{}{}{}{}

\cventry{}{| VCU Symposium for Undergraduate Research}{\textbf{Launch Award for Outstanding Research Poster}}{2015}{}
{
	\begin{adjustwidth}{.25in}{}
		For poster \textit{Evaluation of TCP header fields for data overhead efficiency}.
	\end{adjustwidth}
}

\cventry{2014}{\textbf{Volunteer of the Year} | Grade-school robotics program, Prince William County Schools, VA}{}{}{}{}



%=========================================================================================
%  SECTION: Service
%=========================================================================================
\section{Service}

%==========  Refereeing  =======================================================

\cvitem{Journal reviewer}{\textit{Quantum} (2024, 2022, 2020)}

\cvitem{PC Member}{YQIS 2021}

\cvitem{Conference subreviewer}{STOC 2025, QIP (2025, 2024, 2022), TQC (2023, 2022), ITCS 2023, RANDOM 2023, CCC 2022}

\smallskip
\subsection{Extended commitments (> 1 month)}

\cventry{Spring 2020--Fall 2021}{\textbf{Chair} | UT Graduate Representative Association of Computer Science}{}{}{}
{
	\begin{adjustwidth}{.25in}{}
		\begin{itemize}
			\item GRACS representative to UTCS Diversity, Equity, and Inclusion (DEI) Council.
			\item Co-Organized Graduate Application Assistance Program mentoring under-represented applicants to Ph.D. program. Managed the volunteer mentors. Fall 2020.
		\end{itemize}
	\end{adjustwidth}
}

\cventry{Spring 2019}{\textbf{Tutor} for remedial math students | Manchester High School, Midlothian, VA}{}{}{}
{
	\begin{adjustwidth}{.25in}{}
		Up to 4.5 hours per week with several groups of students.
	\end{adjustwidth}
}

\cventry{}{| VCU Department of Computer Science}{\textbf{Student Advisory Board member}}{(2 academic years) 2016--2018}{}
{
	\begin{adjustwidth}{.25in}{}
		\begin{itemize}
			\item Participated in hiring interviews for new faculty in 2017.
		\end{itemize}
	\end{adjustwidth}
}

\cventry{}{| RamDev: Software Development at VCU}{\textbf{Founder and President}}{(2.5 academic years) 2016--2018}{}
{
	\begin{adjustwidth}{.25in}{}
		\begin{itemize}
			\item Coordinated 46 weekly seminars including 9 corporate speakers and several hackathon trips.
			\item Secured and managed \$2400 in funding and resources.
			\item Increased weekly attendance to 20+ students, becoming largest C.S. organization at VCU.
		\end{itemize}
	\end{adjustwidth}
}

\cventry{Fall 2016}{\textbf{Mentor} | VCU Honors College freshman mentorship program}{}{}{}{}

\cventry{}{| Prince William County Schools, VA}
{\textbf{Volunteer} for grade school robotics competitions (FIRST, Vex robotics)}
{2011--2015}{}
{
	\begin{adjustwidth}{.25in}{}
		\begin{itemize}
			\item Awarded ``Volunteer of the Year'', 2014.
		\end{itemize}
	\end{adjustwidth}
}

\cventry{}{| Wilder Middle School, Richmond, VA}{\textbf{Mentor} for middle School robotics team (FIRST robotics)}{Fall 2014}{}{}

\smallskip
\subsection{Short-term commitments (< 1 month)}

\cventry{Fall 2020}{\textbf{Ph.D. application reviewer} | UT CS Graduate Admissions Committee}{}{}{}{}

\cventry{Spring 2020, Spring 2021}{\textbf{Committee Member} | UT CS GradFest (admitted Ph.D. visit day)}{}{}{}{}

\cventry{Spring 2016, Spring 2017}{\textbf{Lead Dossier Reader} | VCU Honors College graduation dossiers}{}{}{}
{
	\begin{adjustwidth}{.25in}{}
		Assessed dossiers and coordinated other readers.
	\end{adjustwidth}
}

\cventry{2016}{\textbf{Judge} | Launch Award for Outstanding Research Poster}{}{}{}
{
	\begin{adjustwidth}{.25in}{}
		VCU Symposium for Undergraduate Research and Creativity
	\end{adjustwidth}
}


\smallskip
\subsection{Talks and Panels}

\begin{itemize}
	\item[\textbullet] \textbf{Panelist} at Grad school discussion for underrepresented undergraduates. UT CS student organizations, 2020.
	\item[\textbullet] Meeting with U.S. Army Operations Group. I was asked to share my observations from AQIS 2018. November 2018.
	\item[\textbullet] \textbf{Talk}: Computer Science theory \emph{is} fun.\quad VCU RamDev software development club, 2018.
	\item[\textbullet] \textbf{Panelist} at Career workshop for freshman mentorship program. VCU Department of Computer Science, 2017.
	\item[\textbullet] \textbf{Panelist} at Undergraduate conference preparation workshops. VCU Honors College, 2017.
	\item[\textbullet] \textbf{Talk}: Quantum programming (e.g. IBM Q, LIQ$Ui|\rangle$).\quad VCU RamDev software development club, 2017.
\end{itemize}



\end{document}