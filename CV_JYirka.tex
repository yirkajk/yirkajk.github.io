\documentclass[11pt,letterpaper,serif]{moderncv}
\usepackage[margin=0.8in]{geometry}
\moderncvstyle{banking}
\moderncvcolor{black}

\urlstyle{same} % Forces URLs to use same font, rather than a monospace font
\usepackage{changepage} % For indenting blocks of text

\name{Justin}{Yirka} % Unused, since we make our own header

\usepackage{lastpage} % Footer
\fancypagestyle{firstpage}{\rfoot{\textit{\today \quad \thepage / \pageref{LastPage}}}}
\rfoot{\textit{Justin Yirka \quad \thepage / \pageref{LastPage}}} % Any pages without special style will just get this

\renewcommand*{\subsectionfont}{\large\upshape\fontseries{b}\selectfont} % Overrides font for subsection headers

\usepackage{xpatch} % Remove auto formatting from cventry. 
% This is for moderncvbodyiii.sty.
\xpatchcmd{\cventry}{\itshape #2}{#2}{}{} % year
\xpatchcmd{\cventry}{\itshape #3}{#3}{}{} % "title"
\xpatchcmd{\cventry}{\bfseries #4}{#4}{}{} % "institution"
\xpatchcmd{\cventry}{\bfseries #5}{#5}{}{} % city
% #6 "grade" entry is covered by fix to #3 "title"


%=========================================================================================
%  BEGIN DOCUMENT
%=========================================================================================
\begin{document}
	
\hypersetup{pdftitle={Justin Yirka --- CV}, pdfauthor={Justin Yirka}}
	
\thispagestyle{firstpage}

\begin{center}
	{\huge\textbf{Justin Yirka}}
	
	B.S. in Computer Science and B.S. in Mathematics
	
	Virginia Commonwealth University, Richmond, VA, USA
	
	\vspace{\baselineskip}
	
	\href{mailto:yirkajk@vcu.edu}{YirkaJk@vcu.edu}
	
	(703) 229-7956
	
	\url{www.JustinYirka.com}	
\end{center}


%%% Increase vertical spacing after paragraphs in remainder of document (default parskip in moderncv is 0em) %%%
\setlength{\parskip}{0.3em}


%=========================================================================================
%  SECTION: Research Interests
%=========================================================================================
\section{Research Interests}
Quantum computing: complexity theory, algorithms, and connections to applications


%=========================================================================================
%  SECTION: Education
%=========================================================================================
\section{Education}
\cventry{May 2018}{B.S. in Computer Science}{\bfseries Virginia Commonwealth University (VCU)}{}{}{}
\vspace{-\baselineskip}\vspace{-2\parskip}
\cventry{Concurrent/Dual degrees}{B.S. in Mathematical Sciences}{}{}{GPA: 3.98 out of 4.0}
{	
	\begin{adjustwidth}{.25in}{}
		Specialization in Data Science \newline
		Concentration in Pure Math \newline 
		Minor in Physics \newline
		University Honors
	\end{adjustwidth}
}

%=========================================================================================
%  SECTION: Research
%=========================================================================================
\section{Research}
%==========  Experience  =======================================================
\subsection{Experience}
\cventry{}{University of Paderborn, Germany}{Visiting Researcher}{(3 weeks) November 2018}{}
{	
	\begin{adjustwidth}{.25in}{}
		Collaboration with Sevag Gharibian, Ph.D. \newline
		Topic: Complexity theory and algorithms. Worked to show $\textup{QMA}_{1}$-hardness of the quantum satisfcation problem ($k$-QSAT) given qudits of lower dimensions (i.e. improving on current necessary dimensions).
	\end{adjustwidth}
}

\cventry{}{Graph Theory Computational Discovery Lab, VCU}{Research Assistant}{Summer 2018}{}
{	
	\begin{adjustwidth}{.25in}{}
		Supervisor: Craig Larson, Ph.D. \newline
		Topic: Automated conjecturing and graph Hamiltonicity. \textit{Sage, Python}, and \textit{GitHub}. Implemented algorithms for graph properties, improved open-source project structure, and tested graph conjectures.
	\end{adjustwidth}
}

\cventry{}
{\small Joint Center for Quantum Information and Computer Science (QuICS), University of Maryland}
{NSF Research Experience for Undergraduates (REU) / Undergraduate Researcher}{Summer 2017}{}
{	
	\begin{adjustwidth}{.25in}{}
		Supervisor: Andrew Childs, Ph.D. \newline
		Topic: Quantum tomography. Investigated minimum number of Pauli observables necessary to identify a pure state. Attempted to apply numerical results, group theory (i.e. Clifford group), and hypergraph theory.
	\end{adjustwidth}
}

\cventry{}{Quantum Computing Lab, VCU}{Undergraduate Research Assistant}{2015--2016}{}
{
	\begin{adjustwidth}{.25in}{}
		Supervisor: Sevag Gharibian, Ph.D. \newline
		Topic: Quantum computational complexity. Studied quantum oracle classes  (e.g. {\scriptsize $\textup{P}^{\textup{QMA[log]}}$}) characterized by local physical problems and helped develop ``quantum PH'' and ``quantum Toda's Theorem'' ({\scriptsize $\textup{QCPH}\subseteq \textup{P}^{\textup{PP}^{\textup{PP}}}$}).
	\end{adjustwidth}
}


%==========  Preprints  =======================================================
\subsection{Preprints}
{ \leftskip 0.2in \parindent -0.2in % Hanging indent

S. Gharibian, S. Piddock, and J. Yirka\ Oracle complexity classes and local measurements on physical Hamiltonians. Preprint available on arXiv soon. 

S. Gharibian, M. Santha, A. Sundaram, and J. Yirka. Quantum generalizations of the
polynomial hierarchy with applications to QMA(2). Available at \url{https://arxiv.org/abs/1805.11139}.
April 2018.

} % End hanging indent


%==========  Journal Publications  =======================================================
\subsection{Journal Publications}
{ \leftskip 0.2in \parindent -0.2in % Hanging indent

S. Gharibian and J. Yirka. The complexity of simulating local measurements on quantum systems. Accepted to \textit{Quantum} pending revisions to presentation. Available at \url{https://arxiv.org/abs/1606.05626}. 2016.
	
% S. Gharibian and \textbf{J. Yirka}. The complexity of estimating local measurements on quantum systems. \textit{Quantum} Vol (No.): pg--pg, year.
	
}


%%==========  Conference Presentations  =======================================================
\subsection{Conference Presentations \parbox[b][][b]{2em}{\subsectionrule}{\footnotesize (grouped by paper)} } 
% No hanging indent

S. Gharibian, S. Piddock, and J. Yirka. Oracle complexity classes and local measurements on physical Hamiltonians. 
\begin{adjustwidth}{.25in}{} 
	\begin{itemize}
		\item \textbf{Poster by J. Yirka} at 22nd Conference on Quantum Information Processing (QIP). Boulder, CO, USA, Jan. 2019.
		\item \textbf{Contributed talk by J. Yirka} at 18th Asian Quantum Information Science Conference (AQIS). Nagoya, Japan, Sept. 2018.
	\end{itemize}
\end{adjustwidth}

S. Gharibian, M. Santha, A. Sundaram, and J. Yirka. Quantum generalizations of the polynomial hierarchy with applications to QMA(2).
\begin{adjustwidth}{.25in}{} 
	\begin{itemize}
		\item Poster by A. Sundaram at 22nd Conference on Quantum Information Processing (QIP). Boulder, CO, USA, Jan. 2019.
		\item Contributed talk by A. Sundaram at 18th Asian Quantum Information Science Conference (AQIS). Nagoya, Japan, Sept. 2018. --- \textbf{``Long''/plenary talk: top 7\% of submissions}.
		\item Contributed talk by A. Sundaram at 43rd International	Symposium on Mathematical Foundations of Computer Science (MFCS). Liverpool, UK, Aug. 2018.
	\end{itemize}
\end{adjustwidth}

S. Gharibian and J. Yirka. The complexity of simulating local measurements on quantum systems.
\begin{adjustwidth}{.25in}{}
	\begin{itemize}
		\item Contributed talk by S. Gharibian at 12th Conference on the Theory of Quantum Computation, Communication,
		and Cryptography (TQC). Paris, France, 2017.
		\item \textbf{Poster by J. Yirka} at 20th Conference on Quantum Information Processing (QIP). Seattle, USA, 2017. Presented under a different title.
	\end{itemize}
\end{adjustwidth}

J. Yirka. Evaluation of TCP header fields for data overhead efficiency.
\begin{adjustwidth}{.25in}{}
	\begin{itemize}
		\item \textbf{Poster by J. Yirka} at 30th National Conference on Undergraduate Research (NCUR). Asheville, NC, USA, 2016.
		\item \textbf{Poster by J. Yirka} at VCU Symposium for Undergraduate Research and Creativity. Richmond, VA, USA, 2015. --- \textbf{Awarded ``Launch Award for Outstanding Research Poster''}.	
	\end{itemize}
\end{adjustwidth}


%%==========  Departmental Seminars  =======================================================
\subsection{Departmental Seminars}
{ \leftskip 0.2in \parindent -0.2in % Hanging indent
	
Pure state tomography with Pauli observables. QuICS, University of Maryland. 2017.

Quantum complexity of estimating local physical quantities. Department of Computer Science, VCU. 2016. --- \textbf{Only undergraduate invited in previous 5 years}.

}


%==========  Independent Studies  =======================================================
\subsection{Independent Studies}
\cventry{}{VCU {\footnotesize (CMSC 492/CMSC 601)}}{Convex Optimization}{Fall 2017}{}
{
	\begin{adjustwidth}{.25in}{}
		Studied material for graduate optimization course as an undergraduate. Supervised by S. Gharibian.
	\end{adjustwidth}
}


%=========================================================================================
%  SECTION: Scholarships
%=========================================================================================
\section{Scholarships \hspace{2em}{\footnotesize (all dollar amounts in USD)}} 
\cventry{}{\$110,000}{VCU Presidential Scholarship}{2014--May 2018}{Virginia Commonwealth University}
{	
\begin{adjustwidth}{.25in}{}
	Top scholarship offered. Full cost of 4-year tuition, room, and board. \newline
	Awarded to 0.6\% of students
\end{adjustwidth}
}

\cventry{}{\$80,000}{WPI Presidential Scholarship \textmd{[unable to accept]}} {2014}{Worcester Polytechnic Institute}{}

\cventry{}{\$100,000}{Rensselaer Medal Merit Scholarship \textmd{[unable to accept]}} {2014}{Rensselaer Polytechnic Institute}{}

%=========================================================================================
%  SECTION: Funding
%=========================================================================================
\section{Funding}
\cventry{}{\$400}{Travel grant to attend QIP 2019 in Boulder, CO, USA}{January 2019}{QIP student support / NSF}{}

\cventry{}{\$1,900}{Event grants for seminar series by VCU RamDev software development club}{2016--May 2018}{VCU Student Government Association}{}

\cventry{}{\$500}{Travel grant to present at QIP 2017 in Seattle, USA}{2017}{VCU Honors College}{}

\cventry{}{\$550}{Travel grant to present at NCUR 2016 in Asheville, NC, USA}{2016}{VCU Honors College}{}


%=========================================================================================
%  SECTION: Awards and Honors
%=========================================================================================
\section{Awards and Honors}
\cventry{}{VCU College of Humanities and Sciences}{Pure Mathematics Award}{May 2018}{}
{
	\begin{adjustwidth}{.25in}{}
		Awarded to student in pure mathematics concentration with highest graduating GPA.
	\end{adjustwidth}
}

\cventry{}{VCU School of Engineering}{Mark A. Sternheimer Capstone Design Award}{2017}{}
{
	\begin{adjustwidth}{.25in}{}
		For ``innovation and entrepreneurship'' of senior project developing mobile app. \newline
		Included grant of \$660.
	\end{adjustwidth}
}

\cventry{}{Virginia Commonwealth University}{University Student Scholar Award}{2015}{}{}

\cventry{}{VCU Symposium for Undergraduate Research and Creativity}{Launch Award for Outstanding Research Poster}{2015}{}
{
	\begin{adjustwidth}{.25in}{}
		For poster \textit{Evaluation of TCP header fields for data overhead efficiency}.
	\end{adjustwidth}
}

\cventry{}{Grade-school robotics program, Prince William County Schools, VA}{Volunteer of the Year}{2014}{}{}


%=========================================================================================
%  SECTION: Teaching Experience
%=========================================================================================
\section{Teaching Experience}
%==========  VCU  =======================================================
\subsection{VCU}
\cventry{}{Algebra with Applications {\footnotesize (MATH 141)}}{Teaching Assistant}{(2 semesters) 2016--2017}{}
{
	\begin{adjustwidth}{.25in}{}
		Assisted with in-class exercises, offered tutorials, graded assignments. \newline
		Average student evaluation scores --- Fall 2016: 4.78 / 5.0; Spring 2017: 4.36 / 5.0.
	\end{adjustwidth}
}

\cventry{}{Honors College freshman mentorship program}{Mentor for 1\textsuperscript{st} year student}{Fall 2016}{}{}

\cventry{}{Honors Rhetoric {\footnotesize (HONR 200)} --- first-year honors writing and research course}{Teaching Assistant}{Fall 2015}{}{}

%==========  Other  =======================================================
\subsection{Other}
\cventry{}{CPR and first-aid courses for lifeguards}{Instructor}{2016--March 2018}{}{Department of Parks and Recreation, Prince William County, VA}


%=========================================================================================
%  SECTION: Service
%=========================================================================================
\section{Service}
%==========  University Service  =======================================================
\subsection{University Service}
\cventry{}{VCU Department of Computer Science}{Student Advisory Board member}{2016--May 2018}{}
{
	\begin{adjustwidth}{.25in}{}
		\begin{itemize}%
			\item Participated in hiring interviews for new faculty, 2017 (one of two students to participate).
			\item Invited to School of Engineering strategic planning retreat, 2017 (only C.S. undergraduate).
		\end{itemize}
	\end{adjustwidth}
}

\cventry{}{VCU Honors College}{Senior Reader for Honors graduation dossiers}{(2 academic years) 2016--2017}{}
{
	\begin{adjustwidth}{.25in}{}
		Assessed essays submitted in fulfillment of University Honors. Coordinated other readers.
	\end{adjustwidth}
}

\cventry{}{VCU Department of Computer Science}{Panelist --- Career workshop for freshman mentorship program}{2017}{}{}

\cventry{}{VCU Honors College}{Panelist --- Undergraduate conference preparation workshops}{2017}{}{}

\cventry{}{VCU Symposium for Undergraduate Research and Creativity}{Judge --- Launch Award for Outstanding Research Poster}{2016}{}{}

\cventry{}{Major League Hacking (MLH) and VCU Department of Computer Science}{Organizer --- Local Hack Day of Richmond, VA}{2016}{}
{
	\begin{adjustwidth}{.25in}{}
		Hosted event for 30 students, including 12 high school students.
	\end{adjustwidth}
}

%==========  Extracurricular service  =======================================================
\subsection{Extracurricular Service}
\cventry{}{RamDev: Software Development at VCU}{Founder and President}{2016--May 2018}{}
{
	\begin{adjustwidth}{.25in}{}
		\begin{itemize}
			\item Coordinated 46 weekly seminars including 9 corporate speakers.
			\item Secured and managed \$2400 in funding and resources.
			\item Increased weekly attendance to 20 students, becoming largest C.S. organization at VCU.
		\end{itemize}
	\end{adjustwidth}
}

%==========  Community involvement  =======================================================
\subsection{Community Service and Outreach}
\cventry{}
{\upshape{Answered questions about my observations from AQIS 2018.}}
{Asked to meet with U.S. Army Operations Group}{November 2018}{}{}

\cventry{}{VCU RamDev software development club}{Talk --- Computer Science theory \emph{is} fun}{April 2018}{}{}

\cventry{}{VCU RamDev software development club}{Talk --- Quantum programming (e.g. IBM Q, LIQ$Ui|\rangle$)}{2017}{}{}

\cventry{}{Prince William County Schools, VA}
{Volunteer for grade school robotics competitions (FIRST, Vex robotics)}
{2011--2015}{}
{
	\begin{adjustwidth}{.25in}{}
		Awarded ``Volunteer of the Year'', 2014.
	\end{adjustwidth}
}

\cventry{}{Wilder Middle School, Richmond, VA}{Mentor to middle school robotics team (FIRST robotics)}{2014}{}{}

\end{document}