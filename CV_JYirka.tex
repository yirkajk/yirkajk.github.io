\documentclass[11pt,letterpaper,serif]{moderncv}
\usepackage[margin=1in]{geometry}
\moderncvstyle{banking}
\moderncvcolor{black}

\urlstyle{same} % Forces URLs to use same font, rather than a monospace font
\usepackage{changepage} % For indenting blocks of text

\name{Justin}{Yirka} % Unused, since we make our own header

\usepackage{lastpage} % Footer
\fancypagestyle{firstpage}{\rfoot{\today \quad \thepage / \pageref{LastPage}}}
\rfoot{Justin Yirka \quad \thepage / \pageref{LastPage}} % Any pages without special style will just get this

\renewcommand*{\subsectionfont}{\large\upshape\fontseries{b}\selectfont} % Overrides font for subsection headers. Make them bold.

\usepackage{xpatch} % Remove auto formatting from cventry.
% This is for moderncvbodyiii.sty.
\xpatchcmd{\cventry}{\itshape #2}{#2}{}{} % year
\xpatchcmd{\cventry}{\itshape #3}{#3}{}{} % "title"
\xpatchcmd{\cventry}{\bfseries #4}{#4}{}{} % "institution"
\xpatchcmd{\cventry}{\bfseries #5}{#5}{}{} % city
% #6 "grade" entry is covered by fix to #3 "title"

\usepackage{enumitem} % for defining itemsep as needed. Default is 0.
\setlist{itemsep=0em} % default
\newcommand{\pubItemSep}{0.3em}

\newcommand{\todo}[1]{\textcolor{red}{\textbf{(To Do: } #1\textbf{)}}}

%=========================================================================================
%  BEGIN DOCUMENT
%=========================================================================================
\begin{document}

\hypersetup{pdftitle={Justin Yirka --- CV}, pdfauthor={Justin Yirka}}

\thispagestyle{firstpage}

\begin{center}
	{\huge\textbf{Justin Yirka}}

	Ph.D. Student in Computer Science

	The University of Texas at Austin, USA

	\vspace{\baselineskip}

	\href{mailto:yirka@utexas.edu}{yirka@utexas.edu}


	\url{www.JustinYirka.com}

	(703) 229-7956
\end{center}


%%% Increase vertical spacing after paragraphs in remainder of document (default parskip in moderncv is 0em) %%%
\setlength{\parskip}{0.4em}
% itemsep is defined separately

%=========================================================================================
%  SECTION: Research Interests
%=========================================================================================
\section{Research Interests}
Quantum computing \& Theoretical computer science:

\hspace{.25in} Complexity theory, algorithms, and connections to applications


%=========================================================================================
%  SECTION: Education
%=========================================================================================
\section{Education}
\cventry{2019 -- present}{Ph.D. in Computer Science}{\bfseries University of Texas at Austin (UT)}{}{}
{
	\normalsize
	\begin{adjustwidth}{.25in}{}
		Advisor: Scott Aaronson, Ph.D.
	\end{adjustwidth}
}
\vspace{-\baselineskip}\vspace{-\parskip}
\cventry{2022}{M.S. in Computer Science}{}{}
{%
}
{}

\cventry{2018}{B.S. in Computer Science}{\bfseries Virginia Commonwealth University (VCU)}{}{}{}
\vspace{-\baselineskip}\vspace{-2\parskip}
\cventry{Concurrent/Dual degrees}{B.S. in Mathematical Sciences}{}{}
{% GPA: 3.98 out of 4.0
}
{
	\begin{adjustwidth}{.25in}{}
		% Specialization in Data Science \newline
		% Concentration in Pure Math \newline
		Minor in Physics % \newline
		% University Honors
	\end{adjustwidth}
}

%=========================================================================================
%  SECTION: Research
%=========================================================================================
\section{Research}
%==========  Experience  =======================================================
\subsection{Experience}
\cventry{}{Sandia National Laboratories}{R\&D Intern}{Summer 2023--present}{}
{
	\begin{adjustwidth}{.25in}{}
		Supervisors: Ojas Parekh, Ph.D. and John Kallaugher, Ph.D. \newline
		Topic: Hardness of estimating optimum product states of local Hamiltonians. Quantum constrained optimization problems.
	\end{adjustwidth}
}

\cventry{}{Los Alamos National Laboratories Quantum Computing Summer School}{Summer school / Research Assistant}{Summer 2019}{}
{
	\begin{adjustwidth}{.25in}{}
		Supervisor: Yi\u{g}it Suba\c{s}\i, Ph.D. \newline
		Topic: Near-term (NISQ) quantum algorithms. Studied use of qubit resets to construct circuits for entanglement spectroscopy which were noise-resilient \textit{and} low-width.
	\end{adjustwidth}
}

\cventry{}{Graph Theory Computational Discovery Lab, VCU}{Research Assistant}{Summer 2018}{}
{
	\begin{adjustwidth}{.25in}{}
		Supervisor: Craig Larson, Ph.D. \newline
		Topic: Automated conjecturing and graph theory. Studied conditions for graph Hamiltonicity. Assisted with programming and open-source project management.
	\end{adjustwidth}
}

\cventry{}
{\small Joint Center for Quantum Inform. and Computer Science (QuICS), University of Maryland}
{Undergraduate Researcher (NSF REU C.A.A.R.)}{Summer 2017}{}
{
	\begin{adjustwidth}{.25in}{}
		Supervisor: Andrew Childs, Ph.D. \newline
		Topic: Quantum tomography. Investigated minimum number of Pauli observables necessary to identify a quantum pure state.
	\end{adjustwidth}
}

\cventry{}{Quantum Computing Lab, VCU}{Undergraduate Research Assistant}{2015--2016}{}
{
	\begin{adjustwidth}{.25in}{}
		Supervisor: Sevag Gharibian, Ph.D. \newline
		Topic: Complexity theory. Studied quantum oracle classes  (e.g. {\scriptsize $\textup{P}^{\textup{QMA[log]}}$}) and complexity of simulating local measurements. Helped develop a ``quantum PH'' and ``quantum Toda's Theorem'' ({\scriptsize $\textup{QCPH}\subseteq \textup{P}^{\textup{PP}^{\textup{PP}}}$}).
	\end{adjustwidth}
}


%==========  Workshops  =======================================================
\subsection{Invited positions \& Workshops}

\cventry{}{Schloss Dagstuhl --- Quantum Complexity: Theory and Application}{Invitation-only workshop}{June 2021}{}
{
}

\cventry{}{University of Paderborn, Germany}{Visiting Researcher}{November 2018}{}
{
	\begin{adjustwidth}{.25in}{}
		Collaboration with Sevag Gharibian, Ph.D. \newline
		Topic: Complexity theory and algorithms. Studied $\textup{QMA}_{1}$-hardness of the quantum satisfaction problem ($k$-QSAT) given qudits of lower dimensions.
	\end{adjustwidth}
}


%==========  Preprints  =======================================================
\subsection{Preprints}
{ \leftskip 0.2in \parindent -0.2in % Hanging indent

S. Gharibian, S. Piddock, and J. Yirka. Oracle complexity classes and local measurements on physical Hamiltonians. Available at \url{https://arxiv.org/abs/1909.05981}. 2019.

% (Graph theory) N. Bushaw, V. Gupta, C. Larson, S. Loeb, M. Norge, J. Parrish, J. Yirka, and G. Yu. Automated conjecturing and the Hamiltonian problem. In submission. August 2019.

} % End hanging indent


%==========  Journal Publications  =======================================================
\subsection{Journal Publications}
{ \leftskip 0.2in \parindent -0.2in % Hanging indent

S. Gharibian, M. Santha, J. Sikora, A. Sundaram, and J. Yirka. Quantum generalizations of the
polynomial hierarchy with applications to QMA(2). \textit{Computational Complexity}, 31:12, 2022. \href{https://doi.org/10.1007/s00037-022-00231-8}{DOI: 10.1007/s00037-022-00231-8}.

J. Yirka and Y. Subasi. Qubit-efficient entanglement spectroscopy using qubit resets. \textit{Quantum}, 5:535, 2021. \href{https://doi.org/10.22331/q-2021-09-02-535}{DOI: 10.22331/q-2021-09-02-535}.

S. Gharibian and J. Yirka. The complexity of simulating local measurements on quantum systems.
\textit{Quantum}, 3:189, 2019. \href{https://doi.org/10.22331/q-2019-09-30-189}{DOI: 10.22331/q-2019-09-30-189}.

}


%%==========  Conference Presentations  =======================================================
\subsection{Conference Presentations \parbox[b][][b]{2em}{\subsectionrule}{\footnotesize (grouped by paper)} }
% No hanging indent

J. Yirka and Y. Subasi. Qubit-efficient entanglement spectroscopy using qubit resets.
\begin{adjustwidth}{.25in}{}
	\begin{itemize}[itemsep=\pubItemSep]
		\item \textbf{Contributed talk by J. Yirka} at 6th Conference for Young Quantum Information Scientists (YQIS). Virtual, 2021.
		\item Contributed talk by Y. Subasi at APS March Meeting 2021. Virtual.
		\item \textbf{Contributed talk by J. Yirka} at 20th Asian Quantum Information Science Conference (AQIS). Virtual, 2020.
	\end{itemize}
\end{adjustwidth}

S. Gharibian, S. Piddock, and J. Yirka. Oracle complexity classes and local measurements on physical Hamiltonians.
\begin{adjustwidth}{.25in}{}
	\begin{itemize}[itemsep=\pubItemSep]
		\item Contributed talk by S. Piddock at 37th Symposium on Theoretical Aspects of Computer Science (STACS). Montpellier, France, 2020.
		\item \textbf{Contributed talk by J. Yirka} at 23rd Conference on Quantum Information Processing (QIP). Shenzhen, China, 2020.
		\item Poster by S. Piddock at 14th Conference on the Theory of Quantum Computation, Communication, and Cryptography (TQC). College Park, MD, USA, 2019.
		\item Poster by S. Piddock at Workshop on Quantum Computing Theory in Practice (QCTIP). Bristol, UK, 2019.
		\item \textbf{Poster by J. Yirka} at 22nd Conference on Quantum Information Processing (QIP). Boulder, CO, USA, 2019.
		\item \textbf{Contributed talk by J. Yirka} at 18th Asian Quantum Information Science Conference (AQIS). Nagoya, Japan, 2018.
	\end{itemize}
\end{adjustwidth}

S. Gharibian, M. Santha, J. Sikora, A. Sundaram, and J. Yirka. Quantum generalizations of the polynomial hierarchy with applications to QMA(2).
\begin{adjustwidth}{.25in}{}
	\begin{itemize}[itemsep=\pubItemSep]
		\item Poster by A. Sundaram at 22nd Conference on Quantum Information Processing (QIP). Boulder, CO, USA, 2019.
		\item Contributed talk by A. Sundaram at 18th Asian Quantum Information Science Conference (AQIS). Nagoya, Japan, 2018. --- \textbf{``Long''/plenary talk: top 7\% of submissions}.
		\item Contributed talk by A. Sundaram at 43rd International	Symposium on Mathematical Foundations of Computer Science (MFCS). Liverpool, UK, 2018.
	\end{itemize}
\end{adjustwidth}

S. Gharibian and J. Yirka. The complexity of simulating local measurements on quantum systems.
\begin{adjustwidth}{.25in}{}
	\begin{itemize}[itemsep=\pubItemSep]
		\item Contributed talk by S. Gharibian at 12th Conference on the Theory of Quantum Computation, Communication,
		and Cryptography (TQC). Paris, France, 2017.
		\item \textbf{Poster by J. Yirka} at 20th Conference on Quantum Information Processing (QIP). Seattle, USA, 2017. Presented under a different title.
	\end{itemize}
\end{adjustwidth}

J. Yirka. Evaluation of TCP header fields for data overhead efficiency.
\begin{adjustwidth}{.25in}{}
	\begin{itemize}[itemsep=\pubItemSep]
		\item \textbf{Poster by J. Yirka} at 30th National Conference on Undergraduate Research (NCUR). Asheville, NC, USA, 2016.
		\item \textbf{Poster by J. Yirka} at VCU Symposium for Undergraduate Research and Creativity. Richmond, VA, USA, 2015. --- \textbf{Awarded ``Launch Award for Outstanding Research Poster''}.
	\end{itemize}
\end{adjustwidth}


%%==========  Departmental Seminars  =======================================================
\subsection{Departmental Seminars}
{ \leftskip 0.2in \parindent -0.2in % Hanging indent

Pure state tomography with Pauli observables. QuICS, University of Maryland. 2017.

Quantum complexity of estimating local physical quantities. Department of Computer Science, VCU. 2016.
% \textbf{Only undergraduate invited in previous 5 years}

}


%=========================================================================================
%  SECTION: Funding
%=========================================================================================
\section{Scholarships and Funding \hspace{2em}{\footnotesize (all dollar amounts in USD)}}
\cventry{}{\$110,000}{VCU Presidential Scholarship}{2014--2018}{Virginia Commonwealth University}
{
	\begin{adjustwidth}{.25in}{}
		Awarded to 0.6\% of admitted students.
		% Full cost of 4-year tuition, room, and board.
	\end{adjustwidth}
}

\vspace{4\parskip}

\cventry{}{\$1100}{Travel grant to attend QIP 2020 in Shenzhen, China}{2020}{QIP student support / NSF}{}

\cventry{}{\$400}{Travel grant to attend QIP 2019 in Boulder, CO, USA}{2019}{QIP student support / NSF}{}

\cventry{}{\$1,900}{Grants for seminar series by VCU RamDev software development club}{2016--2018}{VCU Student Government Association}{}

\cventry{}{\$660}{Mark A. Sternheimer Capstone Design Award}{2017}{VCU School of Engineering}
{
	\begin{adjustwidth}{.25in}{}
		Grant for developing and testing senior project mobile app.
		% For ``innovation and entrepreneurship
	\end{adjustwidth}
}

\cventry{}{\$500}{Travel grant to present at QIP 2017 in Seattle, USA}{2017}{VCU Honors College}{}

\cventry{}{\$550}{Travel grant to present at NCUR 2016 in Asheville, NC, USA}{2016}{VCU Honors College}{}


%=========================================================================================
%  SECTION: Awards and Honors
%=========================================================================================
\section{Awards and Honors}

\cventry{}{NSF Graduate Research Fellowship Program (NSF GRFP)}{Honorable Mention}{(Awarded twice) 2019, 2020}{}
{
	\begin{adjustwidth}{.25in}{}
		Awarded to top 30\% of over 12,000 applicants.
	\end{adjustwidth}
}

\cventry{}{VCU College of Humanities and Sciences}{Pure Mathematics Award}{May 2018}{}
{
	\begin{adjustwidth}{.25in}{}
		Student in pure math concentration with highest graduating GPA.
	\end{adjustwidth}
}

% \cventry{}{Virginia Commonwealth University}{University Student Scholar Award}{2015}{}{}

\cventry{}{VCU Symposium for Undergraduate Research and Creativity}{Launch Award for Outstanding Research Poster}{2015}{}
{
	\begin{adjustwidth}{.25in}{}
		For poster \textit{Evaluation of TCP header fields for data overhead efficiency}.
	\end{adjustwidth}
}

% \cventry{}{\$80,000}{WPI Presidential Scholarship \textmd{[declined]}} {2014}{Worcester Polytechnic Institute}{}

% \cventry{}{\$100,000}{Rensselaer Medal Merit Scholarship \textmd{[declined]}} {2014}{Rensselaer Polytechnic Institute}{}

\cventry{}{Grade-school robotics program, Prince William County Schools, VA}{Volunteer of the Year}{2014}{}{}


%=========================================================================================
%  SECTION: Teaching Experience
%=========================================================================================
\section{Teaching Experience}
%==========  UT  =======================================================
\subsection{UT}
\cventry{}{Quantum Information Science (Web-based for M.S. program)}{Head Teaching Assistant}{Spring 2022, 2023}{}
{
	\begin{adjustwidth}{.25in}{}
		QIS course for students in online M.S. program.\\
		All lecture content was pre-recorded by S. Aaronson. I was responsible for all other content and logistics, including modifying the homework, exams, and grading for the online format.
		I handled student concerns, academic integrity, and final grades nearly autonomously, with S. Aaronson as instructor of record.\\
		Supervised 4 other teaching assistants.\\
		~\\
		I was tasked with ensuring a successful first iteration of the course for the growing MSCS program at UT. \\
		Spring 2022 course: 200 students, course evaluation 4.1~/~5
	\end{adjustwidth}
}


\cventry{}{Introduction to Quantum Information Science (Honors course)}{Teaching Assistant}{Fall 2021}{}
{
	\begin{adjustwidth}{.25in}{}
	With Scott Aaronson. Taught recitation and graded assignments.
	\end{adjustwidth}
}

\cventry{}{UT International Academy: Software Engineering}{Instructor}{Summer 2021}{}
{
	\begin{adjustwidth}{.25in}{}
		Virtual. Introductory software engineering course for international undergraduate students.
	\end{adjustwidth}
}

%==========  VCU  =======================================================
\subsection{VCU}
\cventry{}{Algebra with Applications {\footnotesize (MATH 141)}}{Teaching Assistant}{(2 semesters) 2016--2017}{}
{
	\begin{adjustwidth}{.25in}{}
		Assisted with daily in-class exercises, offered tutorials, graded assignments. \newline
		Average student evaluation scores --- Fall 2016: 4.78 / 5.0; Spring 2017: 4.36 / 5.0.
	\end{adjustwidth}
}

\cventry{}{Honors College freshman mentorship program}{Mentor for first-year student}{Fall 2016}{}{}

\cventry{}{Honors Rhetoric {\footnotesize (HONR 200)} --- first-year honors writing and research course}{Teaching Assistant}{Fall 2015}{}{}

%==========  Other  =======================================================
\subsection{Other}
\cventry{}{CPR and first-aid courses for lifeguards}{Instructor}{2016--2018}{}{Department of Parks and Recreation, Prince William County, VA}


%=========================================================================================
%  SECTION: Service
%=========================================================================================
\section{Service}

%==========  Refereeing  =======================================================
\subsection{Refereeing}

\cvitem{Journal reviewer:}{\textit{Quantum} (2022, 2020)}

\cvitem{PC Member}{YQIS 2021 --- 6th Conference for Young Quantum Information Scientists}

\cvitem{Conference subreviewer}{QIP (2024, 2022), TQC (2023, 2022), ITCS 2023, CCC 2022}

%==========  Professional Service  =======================================================
\subsection{Professional Service}

\cventry{}{UT Graduate Representative Association of Computer Science (GRACS)}{Representative and Chair}{Spring 2020--Fall 2021}{}
{
	\begin{adjustwidth}{.25in}{}
		\begin{itemize}
			\item GRACS representative to UTCS Diversity, Equity, and Inclusion (DEI) Council, 2020--2021.
		\end{itemize}
	\end{adjustwidth}
}

\cventry{}{UT CS Graduate Admissions Committee}{Ph.D. application reviewer}{Fall 2020}{}
{}

\cventry{}{Graduate Application Assistance Program (GAAP) for UTCS by GRACS}{Lead Mentor}{Fall 2020}{}
{
	\begin{adjustwidth}{.25in}{}
		Student-led program for mentoring under-represented applicants to Ph.D. program. As part of GRACS, I helped organize the first year of this program, managed volunteer mentors, and mentored prospective students.
	\end{adjustwidth}
}

\cventry{}{UT CS student organizations}{Panelist --- Grad school discussion for underrepresented undergraduates}{August 2020}{}{}

\cventry{}{UT Department of Computer Science}{GradFest committee member}{Spring 2020}{}
{
	\begin{adjustwidth}{.25in}{}
		Helped plan visit weekend for admitted Ph.D. students.
	\end{adjustwidth}
}

\cventry{}{\hspace{.25in}I was asked to share my observations from AQIS 2018.}{Met with U.S. Army Operations Group}{November 2018}{}{}

\cventry{}{VCU Department of Computer Science}{Student Advisory Board member}{(2 academic years) 2016--2018}{}
{
	\begin{adjustwidth}{.25in}{}
		Met with department faculty. Participated in hiring interviews for new faculty in 2017.
	\end{adjustwidth}
}

\cventry{}{VCU Honors College}{Senior Reader for Honors program graduation dossiers}{(2 academic years) 2016--2017}{}
{
	\begin{adjustwidth}{.25in}{}
		Coordinated other readers.
	\end{adjustwidth}
}

\cventry{}{VCU Department of Computer Science}{Panelist --- Career workshop for freshman mentorship program}{2017}{}{}

\cventry{}{VCU Honors College}{Panelist --- Undergraduate conference preparation workshops}{2017}{}{}

\cventry{}{VCU Symposium for Undergraduate Research and Creativity}{Judge --- Launch Award for Outstanding Research Poster}{2016}{}{}

%==========  Extracurricular service  =======================================================
\subsection{Extracurricular Service}
\cventry{}{RamDev: Software Development at VCU}{Founder and President}{(2.5 academic years) 2016--2018}{}
{
	\begin{adjustwidth}{.25in}{}
		\begin{itemize}
			\item Coordinated 46 weekly seminars including 9 corporate speakers and several hackathon trips.
			\item Secured and managed \$2400 in funding and resources.
			\item Increased weekly attendance to 20 students, becoming largest C.S. organization at VCU.
		\end{itemize}
	\end{adjustwidth}
}

%==========  Community involvement  =======================================================
\subsection{Outreach and Community Service}

\cventry{}{Manchester High School, Midlothian, VA}{Tutor for remedial math students at local high school}{Spring 2019}{}
{
	\begin{adjustwidth}{.25in}{}
		Up to 4.5 hours per week with several groups of students.
	\end{adjustwidth}
}

\cventry{}{VCU RamDev software development club}{Talk --- Computer Science theory \emph{is} fun}{April 2018}{}{}

\cventry{}{VCU RamDev software development club}{Talk --- Quantum programming (e.g. IBM Q, LIQ$Ui|\rangle$)}{2017}{}{}

\cventry{}{Major League Hacking (MLH) and VCU Department of Computer Science}{Organizer --- Local Hack Day of Richmond, VA}{2016}{}
{
	\begin{adjustwidth}{.25in}{}
		Organized event for 30 students including 12 high school students.
	\end{adjustwidth}
}

\cventry{}{Prince William County Schools, VA}
{Volunteer for grade school robotics competitions (FIRST, Vex robotics)}
{2011--2015}{}
{
	\begin{adjustwidth}{.25in}{}
		Awarded ``Volunteer of the Year'', 2014.
	\end{adjustwidth}
}

\cventry{}{Wilder Middle School, Richmond, VA}{Mentor to middle school robotics team (FIRST robotics)}{Fall 2014}{}{}

\end{document}