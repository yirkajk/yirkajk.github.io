% forked from CV_JYirka at commit 238736b on 5/27/18

\documentclass[11pt,letterpaper,serif]{moderncv}
\usepackage[margin=.75in]{geometry}
\moderncvstyle{banking}
\moderncvcolor{black}

\renewcommand*{\sectionfont}{\normalsize\bfseries\upshape}
\renewcommand*{\subsectionfont}{\normalsize\upshape\fontseries{b}\selectfont}

\urlstyle{same} % Forces URLs to use same font, rather than a monospace font
\usepackage{changepage} % For indenting blocks of text

\usepackage[sorting=yd, style=authortitle,dashed=false, maxnames=99, backend=bibtex]{biblatex} % sort is below. maxnames avoids et al.
\DeclareNameAlias{author}{first-last} % Forces all names to appear in that order. Default of style=authortitle is first author printed surname,name.
\DeclareSortingTemplate{yd}{ % Reverse chronological order, and then whatever order as listed in the .bib file
	\sort{
		\field{presort}
	}
	\sort[final]{
		\field{sortkey}
	}
	\sort[direction=descending]{
		\field{sortyear}
		\field{year}
		\literal{9999}
	}
}
\addbibresource{CV_preprints.bib}
\addbibresource{CV_conferences.bib}
\addbibresource{CV_seminars.bib}
\addbibresource{CV_public.bib}
\nocite{*} % Cite everything in the bib files

\name{Justin}{Yirka} % Unused, since we make our own header

\usepackage{lastpage}
\fancypagestyle{firstpage}{\rfoot{\textit{\today \quad \thepage / \pageref{LastPage}}}}

\rfoot{\textit{Justin Yirka \quad \thepage / \pageref{LastPage}}} % Any pages without special style will just get this


%=========================================================================================
%  BEGIN DOCUMENT
%=========================================================================================
\begin{document}

\hypersetup{pdftitle={Justin Yirka --- Resume},pdfauthor={Justin Yirka}}

\thispagestyle{firstpage}

\begin{center}
	{\huge\textbf{Justin Yirka}}
	
	B.S. in Computer Science and B.S. in Mathematics
	
	\href{mailto:yirkajk@vcu.edu}{YirkaJk@vcu.edu} \hspace{2em} (703) 229-7956
	
	% \url{www.JustinYirka.com} \qquad
	\url{www.linkedin.com/in/yirkajk}	
\end{center}


%%% Increase vertical spacing in remainder of document (default parskip in moderncv is 0em) %%%
\setlength{\parskip}{0em}
\setlength\bibitemsep{\parskip}


%=========================================================================================
%  SECTION: Education
%=========================================================================================
\vspace{-2em}
\section{Education}
\cventry{May 2018}{B.S. in Computer Science}{Virginia Commonwealth University (VCU)}{Richmond, VA}{}{}
\vspace{-1.5em}
\cventry{Dual degrees}{B.S. in Mathematical Sciences}{}{}{GPA: 3.98 out of 4.0}
{	
	\begin{adjustwidth}{.25in}{}
		Specialization in Data Science \newline
		Concentration in Pure Math \newline
		Minor in Physics \newline
		Supported by VCU Presidential Scholarship
	\end{adjustwidth}
}


%=========================================================================================
%  SECTION: Research
%=========================================================================================
\section{Research}
%==========  Experience  =======================================================
\subsection{Experience}
\cventry{Summer 2018}{Research Assistant}{Graph Theory Computational Discovery Lab, VCU}{}{}
{	
	\begin{adjustwidth}{.25in}{}
		Supervisor: Craig Larson, Ph.D. \newline
		Topic: Use of automated conjecturing software (Python) to find conditions for graph hamiltonicity.
	\end{adjustwidth}
}

\cventry{Summer 2017}{NSF REU Undergraduate Researcher}
{
	$\!$\begin{minipage}{0.8\textwidth}
		Joint Center for Quantum Information and Computer Science (QuICS), \newline
		University of Maryland (UMD)	
	\end{minipage}
}
{}{}
{	
	\begin{adjustwidth}{.25in}{}
		Supervisor: Andrew Childs, Ph.D. \newline
		Topic: Quantum tomography. Pure-state tomography with Pauli observables. \newline
		Support: NSF Research Experience for Undergraduates (REU). P.I.: William Gasarch, Ph.D.
	\end{adjustwidth}
}

\cventry{2015--2016}{Undergraduate Research Assistant}{Quantum Computing Lab, VCU}{}{}
{
	\begin{adjustwidth}{.25in}{}
		Supervisor: Sevag Gharibian, Ph.D. \newline
		Topics: Quantum computational complexity. Complexity of local physical problems, quantum oracle classes (e.g. $\textup{P}^{\textup{QMA[log]}}$), quantum variants of the polynomial hierarchy.
	\end{adjustwidth}
}

%==========  Preprints  =======================================================
\printbibliography[heading=subbibliography, title={Preprints}, keyword=preprint]

%%==========  Conference Presentations  =======================================================
\printbibliography[heading=subbibliography, title={Conference Presentations}, keyword=conference]

%%==========  Departmental Seminars  =======================================================
\printbibliography[heading=subbibliography, title={Department Seminars}, keyword=seminar]

%%==========  Public-Audience Talks  =======================================================
\printbibliography[heading=subbibliography, title={Public-Audience Talks}, keyword=public]


%=========================================================================================
%  SECTION: Programming Experience
%=========================================================================================
\section{Programming Experience}
\cvitem{Languages}{Java, C, Python, Sage, Perl, Wolfram Language, Lua}

\cvitem{Software}{LaTeX, git, Unix, Android \& mobile apps, Mathematica, Weka, AutoCAD}

\cvitem{Software Engineering coursework}{Software Engineering (Agile, Android), Algorithm Analysis, Programming Languages (C, Python, Racket), Introduction to Operating Systems, Object Oriented Programming (Java)}

\cvitem{Applications coursework}{Convex Optimization (graduate course), Introduction to Natural Language Processing (assignments in Perl), Introduction to Data Science (Weka), Artificial Intelligence (neural networks), Graphs and Algorithms, Visualization of Physics with Mathematica}

\subsection{Projects}
\cventry{Summer 2018}{Python}
{\href{https://github.com/math1um/objects-invariants-properties}{Graph Brains Project --- Graph Theory Computational Discovery Lab, VCU}}
{}{}
{
	\begin{adjustwidth}{.25in}{}
		Implement functions for calculating graph properties. Manage known examples and properties in Python and SQL. Improve project structure, documentation, and usability.		
	\end{adjustwidth}
}

\cventry{(2 semesters) 2017--May 2018}{Java, Swift, Python, Android, iOS, Raspberry Pi / Unix, Google Firebase}{\textit{Campus Bluetooth tag network} --- Senior project}{}{}
{
	\begin{adjustwidth}{.25in}{}
		Team project developing campus item-tracking system implementing Android, iOS, and Raspberry Pi programs to locate users' items tagged with BLE beacons.
	\end{adjustwidth}
}

\cventry{Fall 2016}{Java, Android, Amazon AWS}{\textit{GeoViewer} Android app --- Software Engineering course project}{}{}
{
	\begin{adjustwidth}{.25in}{}
		Team project with focus on Agile development. Implemented Android app enabling users to share and discover geocached photos.
	\end{adjustwidth}
}

\cventry{2016}{Wolfram Language, Mathematica}{\textit{Run Planner} Mathematica program --- RamHacks hackathon}{}{}
{
	\begin{adjustwidth}{.25in}{}
		Developed program utilizing opensource GPS data to take as input a starting location and a distance goal and output a jogging route of that distance along the city road network.
	\end{adjustwidth}
}

\cventry{2016}{Java, Android}{\textit{GroupMe Stats} Android app --- VTHacks hackathon}{}{}
{
	\begin{adjustwidth}{.25in}{}
		Team project developing app to use GroupMe API to retrieve information about user's GroupMe conversations and provide interesting statistics to the user.
	\end{adjustwidth}
}

\cventry{2010--2014}{C++}
{\href{https://www.vexrobotics.com/vexedr/competition}{Vex}, \href{https://www.firstinspires.org/robotics/ftc}{FIRST}, and \href{http://zerorobotics.mit.edu/}{Zero {\footnotesize (Interational Space Station)}} robotics competitions}
{}{}{}


%=========================================================================================
%  SECTION: Extracurricular Experience
%=========================================================================================
\section{Extracurricular Experience}
\cventry{2016--May 2018}
{\href{https://vcuramdev.github.io/}{RamDev: Software Development at VCU}}
{\href{https://vcuramdev.github.io/}{Founder and President}}
{}{}
{
	\begin{adjustwidth}{.25in}{}
		\begin{itemize}
			\item Coordinated 46 weekly seminars including 9 corporate speakers.
			\item Secured and managed \$2400 in funding and resources.
			\item Increased weekly attendance to 20 students, becoming largest C.S. organization at VCU.
		\end{itemize}
	\end{adjustwidth}
}


%=========================================================================================
%  SECTION: Awards and Honors
%=========================================================================================
\section{Awards and Honors}
\cventry{2014--May 2018}{\$110,000}{Presidential Scholarship}{}{Virginia Commonwealth University}
{	
	\begin{adjustwidth}{.25in}{}
		Top scholarship offered. Full cost of 4-year tuition, room, and board. \newline
		Awarded to 0.6\% of students
	\end{adjustwidth}
}

\cventry{2017}{VCU School of Engineering}
{\href{https://egr.vcu.edu/capstone/sternheimer-awards/}{Mark A. Sternheimer Capstone Design Award}}
{}{}
{
	\begin{adjustwidth}{.25in}{}
		For ``innovation and entrepreneurship'' of senior project developing mobile app. \newline
		Included grant of \$660.
	\end{adjustwidth}
}

\cventry{2015}{VCU Symposium for Undergraduate Research and Creativity}{Launch Award for Outstanding Research Poster}{}{}{}

\cventry{2014}{\$80,000}{Presidential Scholarship \textit{\textmd{[unable to accept]}}} {}{Worcester Polytechnic Institute}{}

\cventry{2014}{\$100,000}{Rensselaer Medal Merit Scholarship \textit{\textmd{[unable to accept]}}} {}{Rensselaer Polytechnic Institute}{}


\end{document}